\chapter*{Technical documentation}
\addcontentsline{toc}{chapter}{Technical documentation}

At the listing \ref{lst:python_kdu_cmd} there are available both ``FILTERS'' dictionary
containing needed options to perform lossless image compression of given input using Kakadu
software and wrapper of running such command from the subshell with the output decoding.
The keys from mentioned before dictionary correspond to their ``pywt'' names. More complex
filters require more verbose command line input. The output of Kakadu is searched with
specific regular expression to retrieve the bitrate of compressed image and then converted to
floating point number. On the other hand, at the listing \ref{lst:python_kdu_script} the
generation process of possible DWT decompositions is depicted. In this example only five
level of DWT chain is calculated. ``V(-)'' stands for vertical decomposition, ``H(-)''
for horizontal and ``B(-:-:-)'' for both. It is worth pointing out that further decomposition
can be applied at the certain level by filling ``-'' in the brackets.

The exemplary command using arbitrary DWT decomposition is shown below. \\
\mintinline{bash}{./bin/Linux-x86-64-gcc/kdu_compress -i ./img/02_Schluesselfelder_Schiff.ppm} \\
\mintinline{bash}{-o ./image.jpx Creversible=yes Clevels=5} \\
\mintinline{bash}{Cdecomp="B(-:-:-),B(-:-:-),B(-:-:-),B(-:-:-),H(-)"}. \\
Other example involves using not Part 1 compliant filter, i.e. Haar \\
\mintinline{bash}{./bin/Linux-x86-64-gcc/kdu_compress -i ./img/02_Schluesselfelder_Schiff.ppm} \\
\mintinline{bash}{-o ./image.jpx Catk=2 Kkernels:I2=R2X2 Clevels=5} \\
\mintinline{bash}{Cdecomp="B(-:-:-),B(-:-:-),B(-:-:-),H(-),V(-)"}.

\begin{listing}[!htb]
\begin{minted}[linenos, breaklines]{python}
import os
import re
import subprocess
from itertools import product
from pathlib import Path


BASE = "./bin/Linux-x86-64-gcc/kdu_compress"
FILTERS = {
    'bior2.2': 'Creversible=yes',
    'haar': 'Catk=2 Kkernels:I2=R2X2',
    'bior2.6': 'Catk=2 Kextension:I2=SYM Kreversible:I2=yes '
               '"Ksteps:I2={4,-1,4,8},{4,-2,4,8}" '
               'Kcoeffs:I2=0.0625,-0.5625,-0.5625,0.0625,'
               '-0.0625,0.3125,0.3125,-0.0625'
}


def run_cmd(path, cmd):
    imgs = f"-i {path} -o ./image.jpx"
    cmd_base = f"{BASE} {imgs}"
    res = subprocess.run(f"{cmd_base} {cmd}",
                         stdout=subprocess.PIPE, shell=True)
    out = res.stdout.decode('utf-8')
    try:
        out = re.search(R"(Layer.+\n\D+)([\d\.]+)", out, re.M).group(2)
    except AttributeError:
        out = "inf"
    return float(out)
\end{minted}
\caption{Python function used to call Kakadu with desired parameters}
\label{lst:python_kdu_cmd}
\end{listing}

\begin{listing}[!htb]
\begin{minted}[linenos, breaklines]{python}
def test_kdu_dwt_comp(path, dwt_filter):
    vertical = ['V(-)']
    horizontal = ['H(-)']
    both = ['B(-:-:-)']
    comb = [vertical, horizontal, both]
    comps = [*product(comb, repeat=5), ]
    cmds = []
    level = "Clevels=5"
    for comp in comps:
        comp = ','.join(x[0] for x in comp)
        cmds.append(f'{FILTERS[dwt_filter]} {level} Cdecomp="{comp}"')
    results = dict()
    for cmd in cmds:
        results[cmd] = run_cmd(path, cmd)
    min_bitrate = min(results, key=results.get)
    with open(f"./ref_results/{Path(path).stem}.txt", "a") as f:
        f.write(f"best: {results[min_bitrate]}\n")


def test_kdu_filter(path):
    results = dict()
    for dwt_filter, cmd in FILTERS.items():
        results[dwt_filter] = run_cmd(path, cmd)
    with open(f"./ref_results/{Path(path).stem}.txt", "a") as f:
        f.write(f"ref: {results['bior2.2']}\n")
    return min(results, key=results.get)


def main():
    for file in os.scandir('./img'):
        print(file.path)
        dwt_filter = test_kdu_filter(file.path)
        test_kdu_dwt_comp(file.path, dwt_filter)
        print("")


if __name__ == '__main__':
    main()
\end{minted}
\caption{Python script used to test various configurations using Kakadu}
\label{lst:python_kdu_script}
\end{listing}

\chapter*{List of abbreviations and symbols}
\addcontentsline{toc}{chapter}{List of abbreviations and symbols}

% discrete cosine transform (DCT)
\begin{itemize}
\item[JPEG] Joint Photographic Experts Group
\item[PNG] Portable Network Graphics
\item[PACSs] Picture Archiving and Communication Systems
\item[DICOM] Digital Imaging and Communications in Medicine
\item[ISO] International Organization for Standardization
\item[DCT] Discrete Cosine Transform
\item[DWT] Discrete Wavelet Transform
\item[SS-DWT] Skipped Steps Discrete Wavelet Transform
\item[HP] Histogram Packing
\item[LL] Low and then low-pass filtered image
\item[LH] Low and then high-pass filtered image
\item[HL] High and then low-pass filtered image
\item[HH] High and then high-pass filtered image
\item[ppi] pixels per inch
\item[JP2] standard JPEG 2000 file extension
\item[J2K] extension used for storing code-stream JPEG 2000 data
\item[MJ2] Motion JPEG 2000
\item[JPWL] JPEG 2000 Wireless
\item[DSP] Digital Signal Processing
\item[SFINAE] Substitution Failure Is Not An Error
\item[RAII] Resource Acquisition Is Initialization
\end{itemize}

\chapter*{Contents of attached CD}
\addcontentsline{toc}{chapter}{Contents of attached CD}

The thesis is accompanied by a CD containing:
\begin{itemize}
\item thesis (\texttt{pdf} file),
\item source code of applications,
\item data sets used in experiments.
\end{itemize}
