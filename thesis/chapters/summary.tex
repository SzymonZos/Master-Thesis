The realized tasks during the all phases of thesis development are as follows.
\begin{itemize}
    \item Initial research in fields of image processing and compression, analysis of JPEG 2000 algorithm. 
    \item More advanced research of algorithms such as DWT, SS-DWT, HP and JPEG 2000 implementation - Kakadu.
    \item Development of basic DWT implementation.
    \item Development of advanced 2D DWT implementation with possibility of skipping transformation of columns or rows.
    \item Setup of Continuous Integration system and implementation of DWT testing component 
    \item Development of initial heuristics allowing to study the effects of DWT modifications compliant
    with the JPEG 2000 standard such as decompositions into sub-bands and usage of different filters
    \item Support of loading and storing both grayscale and color images.
    \item Initial implementation of multi-threaded heuristics.
    \item Conducting preliminary tests and selecting modifications or their variants to be included in the final heuristics.
    \item Development of multi-threaded optimized implementation of final heuristics.
    \item Research on final heuristics - comparison in terms of obtained compression ratio and time with: unmodified JPEG 2000,
    SS-DWT transformation and the transformation determined by an exhaustive search.
\end{itemize}

Moreover, some conclusions were made according to the achieved results. As it was presented
in the \nameref{sec:time_results}, the custom implementation of multithreading achieved
the best performance. The number of worker threads should be closely related to number of
hardware threads available on the specific platform. There are also improvements in terms of
image compression as described in \nameref{sec:results_comparison}. Despite the reference results
being better on average by 33\%, the described solution managed to improve bit rate by 1.5\% in
comparison to JPEG 2000 Part 1 compliant solution. Therefore it can be concluded that the objective
of the thesis has been reached.

Despite being successful in the field of compression improvement in lossless image compression, the best
possible performance was not achieved. Moreover, run-time of described algorithm can also be reduced.
As study shows, compression rate on average was lagging behind the reference at the rate of 33\%. Although
this gap cannot be fully filled, more research on investigating better heuristic’s weights can be developed.
The hypothetical improvement can be also achieved by proposing algorithm that is able to distinguish best
possible filter pair from the other ones.

As it was stated before, the run-time performance of the algorithm and its future extensions can also be improved.
The process of best possible filter pair and discrete wavelet transformation searching can be optimized to perform
more iterative based approach. Currently all possible configuration are generated at the compile time. At the
run-time of application every possible decomposition for selected filter pair is calculated and then compared
with other ones. At the end global minimum of entropy is chosen. This process can be simplified. Instead, each of
possible decomposition at certain discrete wavelet transform step can be evaluated. Therefore, the minimum entropy
could be calculated on the fly and not promising results could be discarded.

The proposed algorithmic development is not as trivial to parallelize as current solution. There are at least
three approaches that are worth to be tested during implementation phase. Firstly, asynchronous primitives from
the standard C++ library such as ``std::async'', ``std::future'' and ``std::promise'' can be employed to perform
this task. This approach leads to the C++ run-time execution determining whether it is beneficial to employ several
threads for certain task. However, as it was shown before in \nameref{sec:time_results}, one can easy be deceived by relying
on the automatic tools that are supposed to make work parallel. The other solution requires employing producer-consumer
architecture to solve described problem. The run-time execution should be able to determine whether more than one
filter pairs can be processed at the same time. Therefore it should be evaluated whether there are hardware threads
available to speed up whole process. Lastly, more sophisticated architecture connected with build custom thread
pools can be employed and benchmarked.

Currently, the implementation of discrete wavelet transform is done in native C++ or even C language. To ensure
the best possible performance this algorithm can be optimized by explicit applying vector instructions depending
on the used CPU. There are two approaches to solve this issue. Firstly, existing implementation which is wrapping
essential function such convolution and downsampling can be used. Secondly, single instruction, multiple data (SIMD)
can be written in-place to leverage the optimal solution. SIMD is a type of parallel computing primitive which stands
for multiple processing of elements performing the same operation on multiple points of data simultaneously. It is
usually part of hardware design and can be directly accessed through the instruction set (ISA). The application can
take advantage of SIMD when the same value is being added from a large number of data points. This is the case
in many multimedia applications.

It is desired not only to compare described earlier solutions but also to profile them at first. This process can
include profiling of generated machine code which is often the case when dealing with compiler intrinsics such as
SIMD functions. For this particular ``llvm-mca'' tool can be used. This application statically measures the performance
of machine code in a specific CPU. It is measured in terms of throughput as well as processor resource consumption.
On the other hand, profiling can include analysis of overall time complexity, frequency and duration of function calls.
Such tools, i.e. profilers use great variety of techniques that aim to collect data. Hardware interrupts, operating
system hooks and performance are included. For this task tools ``callgrind'' (a subprogram of ``valgrind''), ``gprof''
and ``orbit'' can be employed. The first one is able of recording call history among functions in a program's run in
the form of call-graph. Such data consists of number of instructions executed, their relationship to source lines,
caller and callee dependencies among functions and the general number of such calls. The other tools, i.e. ``gprof''
and ``orbit'' (Open Run-time Binary Instrumentation Tool) work in similar fashion.
