\section{Preface}

The usage of digital images is constantly growing across whole world. There are multiple types
of application where memory usage matters to the users. Image compression is a possible solution
to this problem in some of these fields. For example it is mission critical component in medical
and picture archiving and communication systems (PACSs). \cite{entropy} There are two major types
of such compression. The one is lossy variant and the other one is lossless. Applying lossy methods
to the image can result in the occurrence of compression artifacts. However, there are applications
where such disadvantage is negligible, e.g. natural images and photographs processing in Internet day-to-day usage.
\cite{img_compress} On the other hand lossless image compression does not produce such artefacts,
sacrificing some performance and bitrates optimizations. It is employed in mentioned before medical systems.
Images used for the sake of diagnostics can be taken as an example. In some countries there are
regulations that forbid applying lossy compression to such images. \cite{entropy} Moreover, the
usage of lossless variant is more desired when there exists some uncertainty whether information
contained in the image can be discarded. In these scenarios not using any variant of compression
can be the only subsitute of lossless one. \cite{entropy}

Taking into account mentioned before reasons, some compression algorithms have been introduced as ISO
standards. \cite{entropy} Some notable examples of such papers are PNG, JPEG and JPEG 2000 (often written as JP2).
The latter was orginally developed from 1997 to 2000 with the desire of expanding JPEG capabilities.
The main feature of this standard is usage of discrete wavelet transform (DWT) instead of discrete
cosine transform (DCT) which was introduced in the predecessor. \cite{jpeg2000} The other feature
of JPEG 2000 are support for lossy and lossless compression. As can be described before, such compression
is needed to be perfomed in mission critical systems such as medicine. Therefore, the JPEG 2000 standard
is utilized in PACSs and Digital Imaging and Communications in Medicine DICOM standard. \cite{entropy}
This standard consist of 16 ISO parts which contain wide set of features. Some notable ones are core system
coding and its extensions, motion images, testing and reference software. \cite{jpeg2000}

The successor of JPEG standard improved several aspects over its predecessor. With the usage of its algorithms,
e.g. DWT, it was possible to improve compression performance over JPEG. Moreover, there are other improved areas
with even greater importance. The few examples of such features are scalability and editability. \cite{jpeg2000}
The JPEG 2000 standard supports both very low and very high rates of the compression. It comes crucial
in applications that require such flexibility. Another main advantage of this standard is the ability of
effective handling large range of bit rates. It allows to reduce number of steps taken in processing
certain images in comparison to JPEG. As an example, reducing the number of bits in some image below certain
amount using JPEG standard compliant solution requires reducing the resolution of the input at first.
Only after this procedure encoding of the image can be applied. The JPEG 2000 standard supplies adequate feature
named multiresolution decomposition structure which makes such transformation transparent and one step only. \cite{jpeg2000}


\section{Objective of the project}

\begin{itemize}
    \item objective of the thesis 
\end{itemize}


\section{Scope of the thesis}

\begin{itemize}
    \item scope of the thesis
\end{itemize}


\section{Thesis outline}

\begin{itemize}
    \item short description of chapters
    \item clear description of contribution of the thesis's author
\end{itemize}
