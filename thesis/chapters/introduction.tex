The usage of digital images is constantly growing across whole world. There are multiple types
of applications where memory usage matters to the users. Image compression is a possible solution
to this problem in some of these fields. For example it is mission critical component in medical
and picture archiving and communication systems (PACSs) \cite{entropy}. There are two major types
of such compression. The one is lossy variant and the other one is lossless. Applying lossy methods
to the image can result in the occurrence of compression artifacts. However, there are applications
where such disadvantage is negligible, e.g. natural images and photographs processing in Internet day-to-day usage \cite{img_compress}.
On the other hand lossless image compression does not produce such artefacts,
sacrificing some performance and bitrates optimizations. It is employed in mentioned before medical systems.
Images used for the sake of diagnostics can be taken as an example. In some countries there are
regulations that forbid applying lossy compression to such images \cite{entropy}. Moreover, the
usage of lossless variant is more desired when there exists some uncertainty whether information
contained in the image can be discarded. In these scenarios not using any variant of compression
can be the only substitute of lossless one \cite{entropy}.

Taking into account mentioned before reasons, some compression algorithms have been introduced as ISO
standards \cite{entropy}. Some notable examples of such algorithms are PNG, JPEG and JPEG 2000 (often written as JP2).
The latter was originally developed from 1997 to 2000 with the desire of expanding JPEG capabilities.
The main feature of this standard is usage of discrete wavelet transform (DWT) instead of discrete
cosine transform (DCT) which was introduced in the predecessor \cite{jpeg2000}. The other feature
of JPEG 2000 is support for lossy and lossless compression. As described before, such compression
is needed to be performed in mission critical systems such as medicine. Therefore, the JPEG 2000 standard
is utilized in PACSs and Digital Imaging and Communications in Medicine DICOM standard \cite{entropy}.
This standard consist of 16 ISO parts which contain wide set of features. Some notable ones are core system
coding and its extensions, motion images, testing and reference software \cite{jpeg2000}.

The successor of JPEG standard improved several aspects over its predecessor. With the usage of its algorithms,
e.g. DWT, it was possible to improve compression performance over JPEG. Moreover, there are other improved areas
with even greater importance. The few examples of such features are scalability and editability \cite{jpeg2000}.
The JPEG 2000 standard supports both very low and very high rates of the compression. It comes crucial
in applications that require such flexibility. Another main advantage of this standard is the ability of
effective handling large range of bit rates. It allows to reduce number of steps taken in processing
certain images in comparison to JPEG. As an example, reducing the number of bits in some image below certain
amount using JPEG standard compliant solution requires reducing the resolution of the input at first.
Only after this procedure encoding of the image can be applied. The JPEG 2000 standard supplies adequate feature
named multiresolution decomposition structure which makes such transformation transparent and one step only \cite{jpeg2000}.

The standard way of performing discrete wavelet transform (DWT) in the JPEG 2000 compliant with Part 1 is to decompose
the image into sub-bands using a pair of low- and high-pass filters. This decomposition is applied multiple times using
higher DWT orders. The standard order which is used across whole industry is five \cite{jpeg_suite} \cite{jpeg_summary}.
The Part 2 of the standard contains several types of extensions which can be applied to modify the encoding
algorithm. For instance DWT can be modified in a way that makes decomposition of the image into sub-bands of different
shapes possible. Moreover, the strict selection of the pair of filters imposed by Part 1 of the standard can be
broken. However, the same pair has to be used for all sub-bands of the image \cite{jpeg_suite}. The other type of applicable
modification is skipping some steps of discrete wavelet transform (SS-DWT). It is usually beneficial for processing
non-photographic and screen content images. Another way of achieving improvement in terms of compression ratio is
applying the reversible histogram packing. This type of extension significantly improves the ratio of compression
when the histogram of the image is sparse. It means that unused levels appear between frequently used brightness levels.
With the help of described Part 2 compliant extensions to the JPEG 2000 standard it is possible to adaptively
adjust the transform for a specified image to improve the compression ratio. The result of this operation can still
be correctly decoded by every decompressor which is compatible with the Part 2 of the JPEG 2000 standard.

The objective of the thesis is to develop, implement and test several forms of heuristics which can determine
the optimal transform in terms of compression ratio of the given image. Transform shall be compliant with
the Part 2 of JPEG 2000. The heuristics shall be rather fast and use entropy as an estimation of the JPEG 2000 encoding.
Moreover, they can be greedy and use trial and error approach to some extent. The implementation of the program
shall be done in modern C++ to utilize such language capabilities as cross-platform threads. The main target of the
application are multi-core CPU architectures. The result of the project work is a tool that quickly determines 
the transform for the specified image and invokes the JPEG 2000 encoder with selected transform. However, it is acceptable
to achieve small time overhead in terms of the entire compression process. The resulting image shall come with the
improvement of the lossless JPEG 2000 compression ratio.

At the beginning of this paper there is introduction to the domain problem of image processing and compression.
Some methods of applying this kind of compression are described in \fullref{ch:intro}. Moreover, objective and scope
of the thesis are described there. In the \fullref{ch:problem} theory connected to discrete wavelet transform
is presented in details. There are examples related to calculation process of one and two dimensional DWT discussed.
Moreover, introduction to JPEG 2000 technicalities including Part 1 and Part 2 is depicted in that particular chapter.
Lastly, there are scientific paper research and description of existing solutions available. The next chapter, i.e.
\fullref{ch:subject} is strictly connected to the description of proposed solution and implementation details. There are
presented code snippets of the most crucial parts of the application. Another one which is \fullref{ch:experiments}
concentrates on research methodology and final results. Both image compression and time execution of described
solution to the problem are presented.
The last part is \fullref{ch:summary} which wraps up all results and makes some valuable conclusions.
At the end there are appendices available such as technical documentation and list of used tables, listings, etc.
