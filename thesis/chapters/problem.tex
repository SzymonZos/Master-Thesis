\section{Discrete Wavelet Transform}

\subsection{One dimensional DWT}

\begin{figure}
    \centering
    \includegraphics[scale=0.5]{dwt_1d_anal_synth.png}
    \caption{1-D DWT, two-band wavelet analysis and synthesis filter banks \cite{jpeg_suite}}
    \label{fig:dwt_1d_anal_synth}
\end{figure}

The linear convolution (filtering) of sequences $x(n)$ and $h(n)$ is defined as in equation \ref{eq:convolution}:
\begin{equation}
    y(n)=\sum_{m=-\infty}^{\infty}x(m)h(n-m)
\label{eq:convolution}
\end{equation}
The one dimensional discrete wavelet transform can be depicted as successive applications (convolutions) of
one selected pair of high and low-pass filters. The output of such application is then followed
by downsampling by the factor of two. For example, it can be achieved by discarding samples with
odd indices after each of filtering operation. It is better visualized in the Figure \ref{fig:dwt_1d_anal_synth}. \cite{jpeg_suite} 
The pair of low and high-pass filters is known as analysis filter bank in the encoding process.
In the signal decoding process it is featured as a synthesis filter bank. The decoding step requires
using the inverse of discrete wavelet transform. 

Take into consideration a one dimensional signal $x(n) = \{55, 234, 70, 21, 88, 37\}$. It can be better
understood as values of pixels in a part of the grayscale image row. It is followed with a pair of low
and high-pass filters designated by $h_{0}(n)$ and $h_{1}(n)$ respectively. An example of such pair is
a lowpass filter $h_{0}(n) = \{-1, 2, 6, 2, -1\}/8$ and a high-pass filter $h_{1}(n) = \{-1, 2, -1\}/2$. They are both
symmetric and consist of only integer operations. Such pair can be presented in the notation of (5, 3) filter bank.
This convention indicates that the length of lowpass filter is five and the length of high-pass filter is three.
In fact the analysis filter bank presented here was firstly proposed by LeGall and Tabatabai in 1988 and
is used in the JPEG 2000 standard for lossless compression of images. The filtering operation has to
be defined at the signal boundaries. Therefore, the one dimensional signal is extended in both directions.
The Part 1 of the JPEG 2000 standard requires symmetrical extension to be performed in such case. \cite{jpeg_suite}
After applying the required symmetrical padding the signal is extended to
$x(n) = \{21, 70, 234, 55, 55, 234, 70, 21, 88, 37, 37, 88, 21, 70\}$. Then, the low-pass filter is applied
resulting in $x'_{0}(n) = \{197.25, 75.5, 98.375, 67.125, 45.375\}$ and the high-pass one which results in
$x'_{1}(n) = \{44.75, -85.75, 29, 12.75, -29.5\}$.

\begin{figure}
    \centering
    \includegraphics[scale=0.45]{dwt_1d_2_level.png}
    \caption{Computation of a 2-level 4-point DWT using a two-stage two-channel Haar analysis filter bank \cite{dwt_impl}}
    \label{fig:dwt_1d_2_level}
\end{figure}

The next example shows how to compute the two levels of discrete wavelet transform. To speed up the process
no padding option is chosen this time which makes it non-compliant with the JPEG 2000 standard.
The filter used here is the most basic one, i.e. Haar analysis filter bank. It is the first wavelet
from the Daubechies wavelet family. The calculation process is visualized in the Figure \ref{fig:dwt_1d_2_level}. \cite{dwt_impl}

The input is chosen as 4-point signal $X_{\phi}(2, k) = \{2, 1, 3, 4\}$. This notation emphasizes the fact
that it is approximation of the input at scale 2. The so called scaling coefficients (or in other term
approximation at scale 1) $X_{\phi}(1, k)$ are computed by convolving the input $x(k)$ with the low-pass
Haar filter impulse response $l(k) = \{1/\sqrt{2}, 1/\sqrt{2}\}$. In the next step there is downsampling
by a factor of 2 applied. The output of convolution has five values. The middle three from these fives 
correspond to cases where both the given input values overlap with the impulse response. As it was described
earlier, the odd values are preserved in the downsampling process. In a result first and third value of these
three middle ones are the approximation output $X_{\phi}(1, k)$. In the similar way, the detail coefficients
at scale 1 $X_{\psi}(1, k)$ are computed. The input $x(k)$ is convolved with the high-pass filter impulse
response $h(k) = \{-1/\sqrt{2}, 1/\sqrt{2}\}$. Then the downsampling by factor of 2 is performed.
Note that only only approximation output $X_{\phi}(1, k)$ of the first stage goes to the second one.
The $X_{\phi}(0, 0)$ and $X_{\psi}(0, 0)$ are calculated accordingly at the end of the second stage. \cite{dwt_impl}

\subsection{Two dimensional DWT}

The idea of using lowpass filter is the preservation of low frequencies of a signal while trying
to eliminate or at least attenuate the high frequencies. In a result the output signal is the blurred
version of the original one. Therefore, the operating principle of the high-pass filter is completely
opposite. As a result of applying such filter, the high frequencies of the signal are preserved and
the low ones are discarded or at least diminished. The output is a signal consisting of edges, textures
and other details. \cite{jpeg_suite} 

\begin{figure}
    \centering
    \includegraphics[scale=0.7]{dwt_2d_example_wiki.png}
    \caption{2D DWT applied 2 times to an exemplary image \cite{dwt_example_wiki}}
    \label{fig:dwt_2d_example_wiki}
\end{figure}

There is presented an example of the effects of the two dimensional discrete wavelet transform on the Figure \ref{fig:dwt_2d_example_wiki}.
The DWT used here is compliant with the Part 1 of the JPEG2000 standard. The number of DWT stages presented in this  
example is equal to two. Two dimensional discrete wavelet transform applied first time to the original image
yields four same sized sub-images. The LL layer (upper left sub-image) is an approximation of the image and contains the low frequencies.
This layer is once more transformed in the next stage. The LH layer (upper right sub-image) preserves high frequencies from the rows of the image.
As a result vertical lines and details (brightness) can be seen in the produced sub-image. On the other hand, the HL layer (bottom left)
contains high frequencies from the columns of the image. The horizontal details and lines can be noticed there.
Lastly, the HH layer (bottom right) preserves the diagonal lines. \cite{dwt_example_wiki}

\begin{figure}
    \centering
    \includegraphics[scale=0.45]{dwt_2d_1_level.png}
    \caption{Computation of a 1-level 4 $\times$ 4 2-D Haar DWT using a two-stage filter bank \cite{dwt_impl}}
    \label{fig:dwt_2d_1_level}
\end{figure}

The process of computing a 1 level two dimensional discrete wavelet transform with usage of
two-stage analysis Haar filter bank is shown in Figure \ref{fig:dwt_2d_1_level}. Coefficients $\mathbf{X}_{\phi}$
are calculated as a result of lowpass filtering and downsampling to each row of the two dimensional
data. Next, similar process process, i.e. lowpass convolution and downsampling is applied to each column of
resulting data. The rest of coefficients is obtained in very similar fashion to the previous ones.
Coefficients $\mathbf{X}^{H}_{\psi}$ are calculated by applying high-pass filtering and downsampling to each row of the
2-D data $\mathbf{x}$ and then followed by applying sequence of low-pass filtering and downsampling to each
column of the resulting data. Coefficients $\mathbf{X}^{V}_{\psi}$ are obtained by applying low-pass filtering
and then downsampling to each column of the resulting data. Lastly, coefficients $\mathbf{X}^{D}_{\psi}$ are
obtained by applying high-pass filtering and downsampling to each row of the 2-D data $\mathbf{x}$ followed by
applying high-pass filtering and downsampling to each column of the resulting data. In the next stage of more
complex dwt calculating process only the coefficients $\mathbf{X}_{\phi}$ are taken into consideration. \cite{dwt_impl}

\subsection{DWT features summary}

\begin{itemize}
    \item In a nutshell, the discrete wavelet transform is a set of bandpass filters. Usually it is implemented
    with the usage of low and high-pass filters recursively.
    \item The computational complexity of computing the DWT in the best case is linear, i.e. $O(N)$.
    \item The first approach to implement the DWT efficiently is evaluation of the required convolutions
    with the usage of the polyphase filter structure.
    \item The second approach is factorization of the polyphase matrix into a product of a set of sparse matrices.
    \item The two dimensional discrete wavelet transform (with separable filters) is usually computed by the row-column method.
    One dimensional DWT of all the columns is computed at first. Then the 1-D DWT of all the resulting
    rows is calculated. The order of the computation does not matter in terms of achieving the same result.
    \item Additional memory of approximate half the size of the given data is required in the implementation of the DWT.
    \item Data reordering is required for an in-place computation of the DWT.
    \item Data expansion problem can occur due to the finite length of the data in the implementation of the asymmetric filters.
    \item Symmetric filters provide linear phase response and an effective solution to the border problem. \cite{dwt_impl}
\end{itemize}

\section{Part 2 of the JPEG 2000}

% Part2 in details
% write down different filters and basic ones

\subsection{Introduction}

Many ideas have been emerging as the JPEG 2000 was developed. These concept were full of
value-added capabilities. However, they were not that important to be gone through the time-consuming
ISO standardization process. The Part 1 (ISO/IEC, 2004a) of the standard, i.e. Core coding system, 
was originally published in 2000. There was a need to create additional parts to include
missing features. The Part 2 of the standard, published as ISO/IEC 15444-2 or ITU Recommendation
T.801 (ISO/IEC, 2004b), contains multiple such extensions. There is present group of rather small
additions that could not merit entire documents of their own. In the Part 1 Core of JPEG 2000
standard decoders are supposed to handle all of the code-stream functionality. The Part 2
is different from first one in this aspect. It is a collection of options that can be
implemented on demand to meet very specific requirements of the given market. Moreover,
sections within an extension annex can be implemented separately. For example, subsets
of extended file format JPX can be used on their own. Therefore, some features of the Part 2
may be present in the wide spectrum of JPEG 2000 applications while the other ones can be
less common in the decoders. \cite{jpeg_suite}

As it was shown in the previous paragraph, the extensions present in the Part 2 consist of 
very different set of topics that can modify or add some features to the Part 1 JPEG 2000 compliant 
processing chain. Some tools can result in the compression efficiency improvement. Others can
ameliorate the visual appearance of compressed images. Another group of extensions can modify
or extend some functionalities in the other ways. The list of the major topics is presented below. \cite{jpeg_suite}
\newline \newline Compression efficiency:
\begin{itemize}
    \item Variable DC offset (VDCO) - Annex B
    \item Variable scalar quantization (VSQ) - Annex C
    \item Trellis coded quantization (TCQ) - Annex D
    \item Extended visual masking - Annex E
    \item Arbitrary wavelet decomposition - Annex F
    \item Arbitrary wavelet transform kernel - Annexes G and H
    \item Multiple component transform - Annex J
    \item Nonlinear point transform - Annex K \cite{jpeg_suite}
\end{itemize}
\hfill \break Functionalities:
\begin{itemize} 
    \item Geometric manipulation - Annex I
    \item Single-sample overlap (SSO/TSSO) - Annex I
    \item Precinct-dependent quantization - Amendment 1
    \item Extended region of interest - Annex L
    \item Extended file format/metadata (JPX) - Annexes M and N
    \item Extended capabilities signaling - Amendment 2 \cite{jpeg_suite}
\end{itemize}

\subsection{Arbitrary Decomposition}

In the Part 1 of the JPEG 2000 standard there is only one wavelet decomposition structure allowed.
This wavelet is called Mallat dyadic decomposition. Such decomposition is a good first choice
to be applied across a wide spectrum of images. However, other ones can improve the quality of the image
over specialized classes of the applications. The other effect of applying such decompositions are
unequal reductions in the horizontal and vertical dimensional of reduced resolution extracts. \cite{jpeg_suite}

\begin{figure}
    \centering
    \includegraphics[scale=0.45]{part_2_decomp_examples.png}
    \caption{Some examples of decomposition compliant with the Part 2 \cite{jpeg_suite}}
    \label{fig:part_2_decomp_examples}
\end{figure}

Other decomposition styles can be found in the wavelet literature. They include the full packet
tree processing and some of its derivatives. The applied packet decomposition derivatives
can outperform the solution from Part 1 of the JPEG 2000 standard in some applications.
For instance, they come crucial at maintaining regular fine-grain texture. Moreover,
the applications that require processing synthetic aperture radar images can benefit
from using this extension. The US Federal Bureau of Investigation actively uses a 500 ppi
fingerprint compression standard, i.e. WSQ (CJIS, 1997). The decomposition is specialized
for the characteristics of fingerprint imagery at 500 dpi. \cite{jpeg_suite}

Some of these decomposition can be seen of the Figure \ref{fig:part_2_decomp_examples}.
Resolution decomposition is depicted as solid lines. Dashed lines represent extra sublevel
decomposition. On the first example, i.e. image $(a)$, there is available full packet decomposition
with such parameters: NL = 3: Ddfs = 111, Doads = 321, Dsads = all 1s. The next picture
illustrates FBI decomposition wit specified parameters: NL = 5: Ddfs = 11111, Doads = 2321,
Dsads = 11101111111111111. The last image is juts an arbitrary example. \cite{jpeg_suite}

The prespecified decomposition structures are not the only feature of this extension.
Wavelet packet analysis can be also used to design custom decompositions for specific images
or some types of images. It was implemented in Coifman and Wickerhauser, 1992; Ramchandan
anad Vetterli, 1993; Meyer, Averbuch, and Stromberg, 2000. Such applications often start with
a large decomposition tree. Then, they tend to locate a good decomposition based upon
specified optimization metric. \cite{jpeg_suite}

\subsection{Arbitrary Wavelet Transforms}

\begin{table}
    \centering
    \caption{Analysis and synthesis filter taps for the floating-point Daubechies (9, 7) filter bank}
    \label{tab:anal_synth_97i}
\begin{tabular}{ccc}
    \toprule
    n         & Low-pass, $h_{0}(n)$ & Low-pass, $g_{0}(n)$ \\
    \midrule
    $0$       & +0.602949018236360  & +1.115087052457000  \\
    $\pm 1$   & +0.266864118442875  & +0.591271763114250  \\
    $\pm 2$   & -0.078223266528990  & -0.057543526228500  \\
    $\pm 3$   & -0.016864118442875  & -0.091271763114250  \\
    $\pm 4$   & +0.026748757410810  &                     \\
    \bottomrule
\end{tabular}

\bigskip
\bigskip


\begin{tabular}{cc}
    \toprule
    n         & High-pass, $h_{1}(n)$ \\
    \midrule
    $-1$      & +1.115087052457000   \\
    $-2, 0$   & -0.591271763114250   \\
    $-3, 1$   & -0.057543526228500   \\
    $-4, 2$   & +0.091271763114250   \\
              &                      \\
    \bottomrule
\end{tabular}
\quad
\begin{tabular}{cc}
    \toprule
    n        & High-pass, $g_{1}(n)$ \\
    \midrule
    $1$      & +0.602949018236360   \\
    $0, 2$   & -0.266864118442875   \\
    $-1, 3$  & -0.078223266528990   \\
    $-2, 4$  & +0.016864118442875   \\
    $-3, 5$  & +0.026748757410810   \\
    \bottomrule
\end{tabular}
\end{table}

\begin{table}
    \centering
    \caption{Analysis and synthesis filter taps for the integer (5, 3) filter bank}
    \label{tab:anal_synth_53r}
\begin{tabular}{ccc}
    \toprule
    n         & Low-pass, $h_{0}(n)$ & Low-pass, $g_{0}(n)$ \\
    \midrule
    $0$       & +0.75  & +1    \\
    $\pm 1$   & +0.25  & +0.5  \\
    $\pm 2$   & -0.125 &       \\
    \bottomrule
\end{tabular}

\bigskip
\bigskip


\begin{tabular}{cc}
    \toprule
    n         & High-pass, $h_{1}(n)$ \\
    \midrule
    $-1$      & +1   \\
    $-2, 0$   & -0.5 \\
              &      \\
    \bottomrule
\end{tabular}
\quad
\begin{tabular}{cc}
    \toprule
    n        & High-pass, $g_{1}(n)$ \\
    \midrule
    $1$      & +0.75  \\
    $0, 2$   & -0.25  \\
    $-1, 3$  & -0.125 \\
    \bottomrule
\end{tabular}
\end{table}

The Part 1 of the JPEG 2000 standard specifies only two possible wavelet transforms.
The reversible one (5-3R, Table \ref{tab:anal_synth_53r}) and the irreversible one
(9-7I, Table \ref{tab:anal_synth_97i}). As it was stated before, both
are required to perform periodic symmetric signal extension at the boundaries.
It is similar case to the Mallat dyadic decomposition in terms of generic implementation.
These filters can compress quite well a wide set of image types. However, certain image
classes can be compressed more efficiently with other types of wavelets. Such a flexibility
is allowed in the Part 2 compliant applications. The range of wavelet transforms is broadened
to include not only the wider range of whole-sample symmetric ones but also half-sample and
generic nonsymmetric ones. Such ability to handle generic filters makes JPEG 2000 standard
a powerful research tool, together with supporting more than niche compression applications.  \cite{jpeg_suite}

\section{Computer architecture}

% some architectural blah blah, multiscaral, exhausted processor speed up curve -> cores, cores, more cores
    

\section{Known solutions}

\subsection{Part 1 compliant applications}

\begin{itemize}
    \item OpenJPEG
    \item JasPer
    \item Grok
\end{itemize}

\subsection{Kakadu}

Detailed description as it is main reference software.

\subsection{Reversible denoising and lifting based color component transformation}

\subsection{Skipping Selected Steps of DWT Computation}
